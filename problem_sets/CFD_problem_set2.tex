\documentclass[letter,11pt]{article}
\usepackage{authblk}
\usepackage[english]{babel}

%\usepackage{url}
%\usepackage{graphicx}
%\usepackage[boxed,lined]{algorithm2e}
%\usepackage{times}
\usepackage{amsmath}
\usepackage{amssymb}
\usepackage{amsthm}           % wird für proof-Umgebung gebraucht
\usepackage{bm}
\usepackage{enumerate}
\usepackage{empheq}
\usepackage{natbib}
%\usepackage{titlesec}
%\usepackage{anysize}
%\setlength{\textwidth}{160mm}
%\setlength{\oddsidemargin}{2mm}
%\setlength{\textheight}{234mm} % 234
%\setlength{\topmargin}{-20mm} %-5

\usepackage{geometry} % Allows the configuration of document margins
%\geometry{a4paper, textwidth=5.2in, textheight=8.5in,
%  marginparsep=7pt, marginparwidth=.6in} % Document margin settings
\geometry{marginparsep=0pt, marginparwidth=0in,top=1in,bottom=1in, right=1.5in, left=1.5in, headsep=0.1in,headheight=0.5in}

\usepackage{graphicx}  % needed for figures
\usepackage{color}
\usepackage{hyperref}
\hypersetup{
%--- fill inside borders ---
  colorlinks=true,        % false: boxed links; true: colored links
  linkcolor=blue,         % color of internal links
  citecolor=blue,         % color of links to bibliography
}

% set header
\usepackage{fancyhdr}
\pagestyle{fancy}

\fancypagestyle{problem_set}{
\rhead{Fall 2020}
\lhead{Guelph-Waterloo Physics Institute (GWPI)}
\cfoot{\thepage}
%\rfoot{Page \thepage}
}

\renewcommand\Affilfont{\itshape}

\newcommand{\Rn}{\mathbb{R}^n}
\newcommand{\Rm}{\mathbb{R}^m}
\newcommand{\Rtwo}{\mathbb{R}^2}
\newcommand{\Rthree}{\mathbb{R}^3}
\newcommand{\RR}{\mathbb{R}}

\newenvironment{itemize*}%
  {\begin{itemize}%
    \setlength{\itemsep}{0pt}%
    \setlength{\parskip}{0pt}}%
  {\end{itemize}}



\begin{document}
\pagestyle{problem_set}


\phantom{.}
%\vspace{0.1cm}
\begin{center}
{\Large\textbf{Computational Fluid Dynamics\\[0.4cm] Problem Set 2}} \\[0.7cm]
Daniel M. Siegel \\[0.1cm]
Perimeter Institute for Theoretical Physics\\
Department of Physics, University of Guelph
\end{center}

\vspace{1cm}

{\noindent\large\textbf{Problem 1 (Viscous hydrodynamics)}}\\[0.1cm]

\noindent In analogy to Problem 3 $(b)$ of Problem Set 1, derive separate evolution equations for the total kinetic energy density $\frac{1}{2}\rho v^2$ and internal energy density $e=\epsilon\rho$ for a viscous fluid. This is to show that viscous effects dissipate kinetic energy into heat.

\vspace{1cm}

{\noindent\large\textbf{Problem 2 (The MHD equations)}}\\[-0.4cm]

\begin{itemize}
  	\item[$(a)$] (\emph{No-monopole constraint})\\
  		The MHD equations consist of evolution equations and the constraint equation $\nabla\cdot \mathbf{B}=0$. Show that this condition must be satisfied during evolution, i.e., that
  		\begin{equation} 
    		\frac{\partial}{\partial t}(\nabla\cdot\mathbf{B}) = 0.
		\end{equation}
		Guaranteeing this constraint is a tricky problem for numerical codes.

	\item[$(b)$] (\emph{Conservation form of the resistive MHD equations})\\
    	Show that the resistive MHD evolution equations can be written in conservation form:
   		\begin{equation}
      		\mathbf{u}_t + \mathbf{f}^i(\mathbf{u})_{x_i} = \mathbf{S}, \label{eq:Euler_prim}
    	\end{equation}
    	where $\mathbf{u}=(\rho, \mathbf{s}, \tau, \mathbf{B})$, $\mathbf{f}^i = (\rho v_i, T_{ij}, U_i, Y_{ij})$, and $\mathbf{S}= (0,\mathbf{0},0,\mathbf{S}_{\rm res})$. Here, $\mathbf{s}=\rho\mathbf{v}$ is the momentum density,
    	\begin{equation}
    		T_{ij} = \rho v_iv_j + \delta_{ij} \left(p + \frac{B^2}{8\pi}\right) - \frac{1}{4\pi}B_iB_j,
    	\end{equation}
    	is the stress tensor,
    	\begin{equation}
    		\tau = \frac{1}{2}\rho v^2 + e +  \frac{B^2}{8\pi} = E + \frac{B^2}{8\pi}
    	\end{equation}
    	is the total energy density,
    	\begin{equation}
    		\mathbf{U} = (E + p)\mathbf{v} + \frac{1}{4\pi}\left(-\mathbf{v}\times \mathbf{B} + \frac{4\pi}{c}\eta_e \mathbf{j}\right) \times \mathbf{B}
    	\end{equation}
    	represents the total energy flow vector, $Y$ is defined by
    	\begin{equation}
    		Y_{ij} = v_iB_j - v_j B_i,
    	\end{equation}
    	and $\mathbf{S}_{\rm res}$ is the resistive source term given by
    	\begin{equation}
    		\mathbf{S}_{\rm res} = \eta_e \nabla^2\mathbf{B} + \frac{4\pi}{c}\mathbf{j}\times \nabla \eta_e.
    	\end{equation}

    \item[$(c)$] (\emph{Dissipation of magnetic energy})\\
    	Note that the total energy density $\tau$ in $(b)$ is strictly conserved, despite the fact that magnetic energy is dissipated. Use energy evolution equations to show what happens to the dissipated magnetic energy density. You will want to look at separate energy evolution equations similar to Problem 1, and you may reuse expressions from $(b)$.

\end{itemize}

%\vfill
\vspace{1cm}
%\newpage
%\phantom{.}

{\noindent\large\textbf{Problem 3 (M1 moment scheme for radiation transport)}}\\

\noindent Consider the equation of radiation transfer,
\begin{equation}
	\frac{1}{c}\frac{\partial I_\nu}{\partial t} + (\mathbf{n\cdot\nabla}) I_\nu = \left[\frac{\partial I_\nu}{\partial t}\right]_{\rm coll},
\end{equation}
where $\mathbf{n}$ is the unit vector in propagation direction of a photon, and $I_\nu = I_\nu(\mathbf{x},\mathbf{n},\nu,t)$ is the radiation intensity. Following the methods introduced when deriving the Euler equations as moment equations of the Boltzmann equation, devise a moment scheme for radiation transport, formulated as equations in conservation form including the first two moments of the equation of radiation transfer. \\
\emph{Hint:} Consider moments of the radiation intensity that are obtained by integrating over all angular directions in momentum space, i.e., that are of the form
\begin{equation}
	M^{(k)}_{i_1\ldots i_k} \equiv \frac{1}{4\pi} \int_{4\pi} I_\nu n_{i_1}\ldots n_{i_k} {\rm d}\Omega.
\end{equation}
The first few moments are usually denoted by $J_\nu \equiv M^{(0)}$, $H_\nu^i\equiv M^{(1)}_{i}$, and $K^{ij}_\nu\equiv M^{(2)}_{ij}$.\\
\emph{Remark:} In analogy to the equation of state for the Euler equations, the radiation moment scheme requires a relation
\begin{equation}
	K^{ij}_\nu = f_\nu^{ij} J_\nu
\end{equation}
to close the system. Here, $f_\nu^{ij}$ denote the generalized \emph{Eddington factors}, which need to be determined using an independent procedure. Note that due to the properties of $K^{ij}_\nu$, $K^{ij}_\nu = K^{ji}_\nu$ and $\sum_i K^{ii}_\nu=1$, there are five Eddington factors to be specified. For optically thick radiation transfer in stellar interiors the so-called \emph{Eddington approximation}
\begin{equation}
	f^{ij}_\nu = 0 \;\;(i\ne j),\mskip20mu f_\nu^{ij} = \frac{1}{3} \;\; (i=j)
\end{equation}
is appropriate.

\end{document}

