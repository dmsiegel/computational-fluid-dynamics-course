\documentclass[letter,11pt]{article}
\usepackage{authblk}
\usepackage[english]{babel}

%\usepackage{url}
%\usepackage{graphicx}
%\usepackage[boxed,lined]{algorithm2e}
%\usepackage{times}
\usepackage{amsmath}
\usepackage{amssymb}
\usepackage{amsthm}           % wird für proof-Umgebung gebraucht
\usepackage{bm}
\usepackage{enumerate}
\usepackage{empheq}
\usepackage{natbib}
%\usepackage{titlesec}
%\usepackage{anysize}
%\setlength{\textwidth}{160mm}
%\setlength{\oddsidemargin}{2mm}
%\setlength{\textheight}{234mm} % 234
%\setlength{\topmargin}{-20mm} %-5

\usepackage{geometry} % Allows the configuration of document margins
%\geometry{a4paper, textwidth=5.2in, textheight=8.5in,
%  marginparsep=7pt, marginparwidth=.6in} % Document margin settings
\geometry{marginparsep=0pt, marginparwidth=0in,top=1in,bottom=1in, right=1.5in, left=1.5in, headsep=0.1in,headheight=0.5in}

\usepackage{graphicx}  % needed for figures
\usepackage{color}
\usepackage{hyperref}
\hypersetup{
%--- fill inside borders ---
  colorlinks=true,        % false: boxed links; true: colored links
  linkcolor=blue,         % color of internal links
  citecolor=blue,         % color of links to bibliography
}

% set header
\usepackage{fancyhdr}
\pagestyle{fancy}

\fancypagestyle{problem_set}{
\rhead{Fall 2020}
\lhead{Guelph-Waterloo Physics Institute (GWPI)}
\cfoot{\thepage}
%\rfoot{Page \thepage}
}

\renewcommand\Affilfont{\itshape}

\newcommand{\Rn}{\mathbb{R}^n}
\newcommand{\Rm}{\mathbb{R}^m}
\newcommand{\Rtwo}{\mathbb{R}^2}
\newcommand{\Rthree}{\mathbb{R}^3}
\newcommand{\RR}{\mathbb{R}}

\newenvironment{itemize*}%
  {\begin{itemize}%
    \setlength{\itemsep}{0pt}%
    \setlength{\parskip}{0pt}}%
  {\end{itemize}}



\begin{document}
\pagestyle{problem_set}


\phantom{.}
%\vspace{0.1cm}
\begin{center}
{\Large\textbf{Computational Fluid Dynamics\\[0.4cm] Problem Set 5}} \\[0.7cm]
Daniel M. Siegel \\[0.1cm]
Perimeter Institute for Theoretical Physics\\
Department of Physics, University of Guelph
\end{center}

\vspace{1cm}

{\noindent\large\textbf{Problem 1 (Project: Riemann solver for the Euler equations)}}\\[0.1cm]

\noindent Implement (in a language of your choice, but excluding Fortran) an `exact' Riemann solver to solve the Riemann problem for the one-dimensional time-dependent Euler equations for ideal gases. To verify your implementation and explore solutions, use an adiabatic constant of $\gamma=1.4$ and consider the following initial data for the primitives $\mathbf{w}=(\rho,v,p)$:\\

\begin{tabular}{c|c|c|c|c|c|c}
  Problem & $\rho_l$ & $v_l$ & $p_l$ & $\rho_r$ & $v_r$ & $p_r$ \\
  \hline
  1 & 1.0 & 0.0 & 1.0 & 0.125 & 0.0 & 0.1 \\
  2 & 1.0 & -2.0 & 0.4 & 1.0 & 2.0 & 0.4 \\
\end{tabular}\\

\noindent Problem 1 results in a left rarefaction wave and a right shock, while Problem 2 results in a two-rarefaction wave solution. Plot density, pressure, velocity and internal energy at time $t = 0.25$ for Problem 1 and $t=0.15$ for Problem 2 on a domain $x\in [-1,1]$ and verify that the behavior of these quantities across the three non-linear wave structures looks as expected.\\

\noindent\emph{Hints:}
\begin{enumerate}
  \item Follow the solution strategy as discussed in the lectures. You will find additional detail in Chap.~4 of Toro's book. You can check your solutions against Tabs.~4.2 and 4.3 in Toro's book.
  
  \item First, find the pressure in the star region $p^*$, using a Newton-Raphson scheme on the pressure function $f(p)$ discussed in the lectures. You may use $p^*_{\rm guess}=\frac{1}{2}(p_l + p_r)$ as initial guess for $p^*$ and a relative tolerance of $10^{-6}$ for the root-finding procedure.

  \item Second, compute the solution for the primitives $\mathbf{w}(x,t)=(\rho(x,t), v(x,t), p(x,t))$. Use a spatial grid $x\in [-1,1]$ with the initial discontinuity at $x=0$. Due to the self-similar nature of the solution, sample the solution for the primitives $\mathbf{w}(x,t)$ as a function of the `speed' $S=x/t$. The corresponding solutions were discussed in the lectures. Once you specify the time $t$, the solution profiles become a function of $x$ only.
\end{enumerate}



%\vspace{2cm}

\newpage

\phantom{.}

{\noindent\large\textbf{Problem 2 (Bonus question---discrete entropy condition)}}\\

\noindent Consider the situation of the Lax-Wendroff theorem and assume that the conservative scheme is consistent with the discrete entropy condition. Show that the scheme then converges to a weak solution of the conservation law that satisfies the entropy condition.\\

\noindent\emph{Hint:} The proof is similar to the proof of the Lax-Wendroff theorem.

\end{document}

