\documentclass[letter,11pt]{article}
\usepackage{authblk}
\usepackage[english]{babel}

%\usepackage{url}
%\usepackage{graphicx}
%\usepackage[boxed,lined]{algorithm2e}
%\usepackage{times}
\usepackage{amsmath}
\usepackage{amssymb}
\usepackage{amsthm}           % wird für proof-Umgebung gebraucht
\usepackage{bm}
\usepackage{enumerate}
\usepackage{empheq}
\usepackage{natbib}
%\usepackage{titlesec}
%\usepackage{anysize}
%\setlength{\textwidth}{160mm}
%\setlength{\oddsidemargin}{2mm}
%\setlength{\textheight}{234mm} % 234
%\setlength{\topmargin}{-20mm} %-5

\usepackage{geometry} % Allows the configuration of document margins
%\geometry{a4paper, textwidth=5.2in, textheight=8.5in,
%  marginparsep=7pt, marginparwidth=.6in} % Document margin settings
\geometry{marginparsep=0pt, marginparwidth=0in,top=1in,bottom=1in, right=1.5in, left=1.5in, headsep=0.1in,headheight=0.5in}

\usepackage{graphicx}  % needed for figures
\usepackage{color}
\usepackage{hyperref}
\hypersetup{
%--- fill inside borders ---
  colorlinks=true,        % false: boxed links; true: colored links
  linkcolor=blue,         % color of internal links
  citecolor=blue,         % color of links to bibliography
}

% set header
\usepackage{fancyhdr}
\pagestyle{fancy}

\fancypagestyle{problem_set}{
\rhead{Fall 2020}
\lhead{Guelph-Waterloo Physics Institute (GWPI)}
\cfoot{\thepage}
%\rfoot{Page \thepage}
}

\renewcommand\Affilfont{\itshape}

\newcommand{\Rn}{\mathbb{R}^n}
\newcommand{\Rm}{\mathbb{R}^m}
\newcommand{\Rtwo}{\mathbb{R}^2}
\newcommand{\Rthree}{\mathbb{R}^3}
\newcommand{\RR}{\mathbb{R}}

\newenvironment{itemize*}%
  {\begin{itemize}%
    \setlength{\itemsep}{0pt}%
    \setlength{\parskip}{0pt}}%
  {\end{itemize}}



\begin{document}
\pagestyle{problem_set}


\phantom{.}
%\vspace{0.1cm}
\begin{center}
{\Large\textbf{Computational Fluid Dynamics\\[0.4cm] Problem Set 4}} \\[0.7cm]
Daniel M. Siegel \\[0.1cm]
Perimeter Institute for Theoretical Physics\\
Department of Physics, University of Guelph
\end{center}

\vspace{1cm}

{\noindent\large\textbf{Problem 1 (Jump conditions for the Euler equations)}}\\[0.1cm]

\noindent Consider a 3-shock with shock velocity $S_3$ for the 1D Euler equations in conservative form, assuming an ideal gas with adiabatic constant $\gamma$. Let $(\rho_\star, v_\star, p_\star)$ and $(\rho_r, v_r, p_r)$ denote the sets of primitive variables on the left and right side of the shock front, respectively. Derive the Rankine-Hugoniot jump conditions
\begin{equation}
  \frac{\rho_\star}{\rho_r} = \frac{\frac{p_\star}{p_r}+ \frac{\gamma-1}{\gamma+1}}{\frac{\gamma-1}{\gamma+1}\frac{p_\star}{p_r} + 1}, \mskip40mu s_3 = v_r + c_r \sqrt{\frac{\gamma + 1}{2\gamma}\frac{p_\star}{p_r}+\frac{\gamma-1}{2\gamma}},
\end{equation}
where $c=\sqrt{\gamma p / \rho}$ denotes the sound speed.

\vspace{2cm}

{\noindent\large\textbf{Problem 2 (Non-uniqueness of weak solutions)}}\\[-0.4cm]

\noindent A given initial value problem for conservation laws admits, in general, infinitely many weak solutions. Consider the Riemann problem for the inviscid Burgers' equation:
\begin{equation}
  \left\{ \begin{array}{l}
  \partial_t u + \partial_x\left(\frac{1}{2}u^2\right) = 0 \\
  u(0,x) = \left\{ \begin{array}{cc}
  1 & \mathrm{if}\, x>0 \\
  0 & \mathrm{if}\, x<0 \\
  \end{array}\right.\\
  \end{array}\right.. \label{eq:RP_Burgers}
  \end{equation}
Show that for any $\alpha\in(0,1)$ the `two-shock solution'
\begin{equation}
  u_\alpha(t,x) \equiv \left\{ \begin{array}{ll}
    0 & \mathrm{if}\, x< \frac{\alpha}{2}t \\
    \alpha & \mathrm{if}\, \frac{\alpha}{2}t<x<\frac{1+\alpha}{2}t \\
    1 & \mathrm{if}\, x>\frac{1+\alpha}{2}t \\
  \end{array}\right. .
\end{equation}
is a weak solution to the Cauchy problem \eqref{eq:RP_Burgers}.

%\vfill
%\vspace{1cm}

\newpage
\phantom{.}

{\noindent\large\textbf{Problem 3 (Entropy pair \& Riemann invariants)}}\\

\noindent Consider the 1D Euler equations for an ideal gas with specific entropy $S = S_0 + c_v\ln\frac{p}{\rho^\gamma}$, where $S_0$ and $c_v$ are constants and $\gamma$ is the adiabatic index.

\begin{enumerate}
  \item \emph{(Entropy pair)} Show that the specific entropy satisfies
  \begin{equation}
    S_t + v S_x = 0
  \end{equation}
  where the solution is smooth. What is the entropy flux? Show that modulo a minus sign the physical entropy provides an entropy--entropy-flux pair for the Euler equations.

  \item \emph{(Riemann invariants)} Consider the Euler equations in primitive form using the primitives $(\rho,v,S)$. Show that $S$ is a Riemann invariant. Show that there exist the following $k$-Riemann invariants:
  \begin{itemize}
    \item $k = 1$: $S$, $u+ \frac{2}{\gamma - 1}c$
    \item $k = 2$: $u$, $p$
    \item $k = 3$: $S$, $u- \frac{2}{\gamma - 1}c$
  \end{itemize}

  \item \emph{(Characteristic fields)} Show that the $k=2$ characteristic field is linearly degenerate, while the $k=1,3$ characteristic fields are genuinely non-linear. \emph{Hint:} It is easiest to work with the conservative formulation here.

\end{enumerate}

\vspace{1cm}

{\noindent\large\textbf{Problem 4 (Solution to Burgers' equation)}}\\

\noindent Determine and sketch the exact entropy solution to Burgers' equation $u_t + (\frac{1}{2}u^2)_x = 0$ for all $t>0$ for initial data given by
\begin{equation}
  u^0(x) = \left\{ \begin{array}{ll}
  -1 & \mathrm{if}\, x<-1 \\
  0 & \mathrm{if}\, -1<x<1 \\
  1 & \mathrm{if}\, x>1
  \end{array}\right.
\end{equation}

\vspace{1cm}

{\noindent\large\textbf{Problem 5 (Bonus question, not required)}}\\

\noindent Consider a piecewise smooth weak solution $u$ of the system $u_t + f(u)_x = 0$, and assume that an entropy pair to the equation is given by $(\Phi,\Psi)$. Show that the entropy condition along the discontinuity with velocity $s$ leads to the inequality
\begin{equation}
  s[\Phi(u_l) - \Phi(u_r)] \le \Psi(u_l) - \Psi(u_r),
\end{equation}
where $u_l$ and $u_r$ denote the values of $u$ immediately left and right of the discontinuity.
\emph{Hint:} Consider the Rankine-Hugoniot theorem.
\emph{Remark:} This is the first step in showing the equivalence of the Lax entropy condition and the entropy inequalities.

\end{document}

