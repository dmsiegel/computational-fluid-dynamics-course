\documentclass[letter,11pt]{article}
\usepackage{authblk}
\usepackage[english]{babel}

%\usepackage{url}
%\usepackage{graphicx}
%\usepackage[boxed,lined]{algorithm2e}
%\usepackage{times}
\usepackage{amsmath}
\usepackage{amssymb}
\usepackage{amsthm}           % proof environment etc.
\usepackage{bm}
\usepackage{enumerate}
\usepackage{empheq}
\usepackage{natbib}
%\usepackage{titlesec}
%\usepackage{anysize}
%\setlength{\textwidth}{160mm}
%\setlength{\oddsidemargin}{2mm}
%\setlength{\textheight}{234mm} % 234
%\setlength{\topmargin}{-20mm} %-5

\usepackage{geometry} % Allows the configuration of document margins
%\geometry{a4paper, textwidth=5.2in, textheight=8.5in,
%  marginparsep=7pt, marginparwidth=.6in} % Document margin settings
\geometry{marginparsep=0pt, marginparwidth=0in,top=1in,bottom=1in, right=1.5in, left=1.5in, headsep=0.1in,headheight=0.5in}

\usepackage{graphicx}  % needed for figures
\usepackage{color}
\usepackage{hyperref}
\hypersetup{
%--- fill inside borders ---
  colorlinks=true,        % false: boxed links; true: colored links
  linkcolor=blue,         % color of internal links
  citecolor=blue,         % color of links to bibliography
}

% set header
\usepackage{fancyhdr}
\pagestyle{fancy}

\fancypagestyle{problem_set}{
\rhead{Fall 2020}
\lhead{Guelph-Waterloo Physics Institute (GWPI)}
\cfoot{\thepage}
%\rfoot{Page \thepage}
}

\renewcommand\Affilfont{\itshape}

\newcommand{\Rn}{\mathbb{R}^n}
\newcommand{\Rm}{\mathbb{R}^m}
\newcommand{\Rtwo}{\mathbb{R}^2}
\newcommand{\Rthree}{\mathbb{R}^3}
\newcommand{\RR}{\mathbb{R}}


\newenvironment{itemize*}%
  {\begin{itemize}%
    \setlength{\itemsep}{0pt}%
    \setlength{\parskip}{0pt}}%
  {\end{itemize}}



\begin{document}
\pagestyle{problem_set}


\phantom{.}
%\vspace{0.1cm}
\begin{center}
{\Large\textbf{Computational Fluid Dynamics\\[0.4cm] Problem Set 1}} \\[0.7cm]
Daniel M. Siegel \\[0.1cm]
Perimeter Institute for Theoretical Physics\\
Department of Physics, University of Guelph
\end{center}

\vspace{1cm}

{\noindent\large\textbf{Problem 1 (Hyperbolic systems)}}%\\[0.1cm]

\begin{itemize}
  \item[$(a)$] This problem illustrates how the term ``hyperbolic'' can be motivated for 1st order systems of PDEs. Show that any linear, constant-coefficient second-order hyperbolic PDE in $\RR^{n+1}$ of the form
  \begin{equation}
    \sum_{i,j = 0}^{n} a_{ij} u_{x_i x_j} + b(\{u_{x_i}\}) = 0
  \end{equation}
   can be recast as a system of 1st order hyperbolic PDEs. \newline
   \emph{Hint:} Note that it suffices to consider an equation of the form $u_{tt}-\Delta u + b(\{u_{x_i}\}) = 0$, since the principal part can always be brought into this form by a suitable coordinate transformation (as shown in the lectures). Here, $t=x_0$ and $\Delta$ only acts on $(x_1,\ldots,x_n)\in\Rn$. You may want to consider a vector $\mathbf{u}$ composed of derivatives $u_{x_i}$.

   \item[$(b)$] Another way to motivate hyperbolicity for systems of $m$ 1st order PDEs,
    \begin{equation}
    \mathbf{u}_t + \sum_{i=1}^{n} A_{i}(x,t) \mathbf{u}_{x_i} = 0,
    \end{equation}
    is to  require the existence of $m$ independent plane wave solutions $\mathbf{u}(\mathbf{x},t) = \mathbf{v}(\mathbf{\xi\cdot x} - \sigma t)$ for any direction $\xi\in\Rn$. Using this ansatz, show that hyperbolicity implies the existence of $m$ independent plane waves.
\end{itemize}

\vspace{1cm}

{\noindent\large\textbf{Problem 2 (Hyperbolicity of Euler's equations)}}\\ %[0.2cm]

\noindent Consider the 1D Euler equations in conservation form,
\begin{equation}
    \mathbf{u}_t + \mathbf{f}(\mathbf{u})_x = 0, \label{eq:Euler_con}
\end{equation}
where $\mathbf{u} = (\rho, \rho v, E)$ and $\mathbf{f} = (\rho v, \rho v^2 + p, v(E+p))$. Assume a general equation of state $p = p(\rho,\epsilon)$. Here, $E=\frac{1}{2}\rho v^2 + e$ is the total energy density, and $\epsilon = e/\rho$ denotes the specific internal energy.

\begin{itemize}
  \item[$(a)$] (\emph{Invariance of hyperbolicity under coordinate transformation}) \\
  Consider $\mathbf{u}$ as a solution to a strictly hyperbolic system of the type
  \begin{equation}
    \mathbf{u}_t + A(\mathbf{u})\mathbf{u}_x = 0,
  \end{equation}
  and let $\Phi: \Rm \rightarrow \Rm$ denote a smooth coordinate transformation with inverse $\Phi^{-1}$. Show that $\tilde{\mathbf{u}}\equiv \Phi(\mathbf{u})$ satisfies the strictly hyperbolic system
  \begin{equation}
    \tilde{\mathbf{u}}_t + \tilde{A}(\tilde{\mathbf{u}})\tilde{\mathbf{u}}_x = 0,
  \end{equation}
  where $\tilde{A}(\mathbf{w})\equiv D\Phi(\Phi^{-1}(\mathbf{w}))A(\Phi^{-1}(\mathbf{w}))D(\Phi^{-1})(\mathbf{w})$ for $\mathbf{w}\in \Rm$.

  \item[$(b)$] (\emph{Conservatives to primitives recovery})\\
    Write down the coordinate transformation $\Phi: \mathbf{u}\mapsto \mathbf{w}$, where $\mathbf{w}\equiv (\rho,v,\epsilon)$ are the primitive variables, and derive the Euler equations in primitive (non-conservation) form
    \begin{equation}
      \mathbf{w}_t + A(\mathbf{w})\mathbf{w}_x = 0. \label{eq:Euler_prim}
    \end{equation}

  \item[$(c)$] (\emph{Strict hyperbolicity})\\
  Starting from \eqref{eq:Euler_prim} and using $(a)$, show that the Euler equations \eqref{eq:Euler_con} are strictly hyperbolic if the local sound speed is positive,
    \begin{equation}
      c_{\rm s} \equiv \left(\frac{p}{\rho^2}\frac{\partial p}{\partial \epsilon} + \frac{\partial p}{\partial\rho}\right)^{1/2} > 0,
    \end{equation}
    i.e., provided that $p>0$ and $\partial p/\partial \epsilon > 0$, $\partial p/\partial\rho > 0$. This shows that numerical schemes that exploit strong hyperbolicity cannot handle vacuum, but instead require at least a tenous ``atmosphere'' on the computational domain.
\end{itemize}

%\vfill
\vspace{1cm}
%\newpage
%\phantom{.}

{\noindent\large\textbf{Problem 3 (Euler equations: energy evolution)}}\\

\begin{itemize}
  \item[$(a)$] From the second moment of the Boltzmann equation, and following the methods introduced in the lectures, derive the equation for the total energy density,
  \begin{equation}
    \frac{\partial E}{\partial t} + \frac{\partial}{\partial x_j}\left[(E+p)v_j\right] = -\frac{\partial h_j}{\partial x_j} + \rho v_j F_j, \label{eq:Euler_E}
  \end{equation}
  where $h_j \equiv \int \frac{m}{2}\tilde{u}_j\tilde{u}^2 f d^3u$ is the heat flux.
  \item[$(b)$]
    Only using the continuity and momentum equations, derive a separate evolution equation for the bulk (kinetic) energy of the fluid,
    \begin{equation}
      \frac{\partial }{\partial t}\left(\frac{1}{2}\rho v^2\right) + \frac{\partial}{\partial x_j}\left(\frac{1}{2}\rho v^2 v_j\right) = -v_j\frac{\partial p}{\partial x_j} + \rho v_j F_j.
    \end{equation}
    Using Eq.~\eqref{eq:Euler_E} and assuming $h_j\equiv 0$ then derive the evolution equation for the internal energy density,
    \begin{equation}
      \frac{\partial e}{\partial t} + \frac{\partial}{\partial x_j}(e v_j) = -p\frac{\partial v_j}{\partial x_j}.
    \end{equation}
  \end{itemize}

\end{document}

